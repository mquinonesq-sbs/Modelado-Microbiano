\documentclass[10pt,conference]{IEEEtran}
\usepackage[utf8]{inputenc}
\usepackage{graphicx}
\usepackage{amsmath}
\usepackage{amsfonts}
\usepackage{siunitx}
\usepackage{hyperref}
\usepackage{caption}
\usepackage{subcaption}
\usepackage{cite}

\title{Modelado y simulacion del crecimiento microbiano mediante automatas celulares bidimensionales}
\author{Codex GPT-5\\Trabajo-Maye}

\begin{document}
\maketitle

\begin{abstract}
Se implemento un automata celular (CA) bidimensional para reproducir el modelo del articulo de referencia. El CA usa una malla de \num{200x200}, vecindario de Moore, umbral de inhibicion espacial \(N_0\), probabilidades de division \(P \in \{0.5, 0.25, 0.125\}\) y un campo de sustrato con difusion y consumo. Se generaron simulaciones batch (300 iteraciones) y animaciones interactivas. Las figuras base (snapshots, curvas por \(N_0\), curvas por \(P\), efecto del sustrato inicial y comparacion cinetica) se producen con \texttt{python main.py} o \texttt{python experiments.py}.
\end{abstract}

\section{Modelo}
Estados: 0 (vacio/muerta), 1 (division), 2 (crecimiento). Malla toroidal \(200\times200\). Vecindario de Moore de 8 celdas. Reglas: inhibicion espacial si \(N_1+N_2 > N_0\) fuerza \(1\to2\); division estocastica \(2\to1\) y colonizacion de huecos \(0\to2\) segun probabilidad y disponibilidad de sustrato; sin muerte explicita (la expansion se frena por espacio y sustrato).

\section{Resultados}
\subsection{Evolucion base}
\begin{figure}[ht]
    \centering
    \includegraphics[width=0.48\textwidth]{figuras/exp_base_snapshots.png}
    \caption{Distribucion espacial a \(t=1,5,10,30\) h para \(N_0=3, P=0.5, s_0=60\,\si{g/L}\). Azul: division, rojo: crecimiento.}
    \label{fig:espacial}
\end{figure}

\begin{figure}[ht]
    \centering
    \includegraphics[width=0.48\textwidth]{figuras/exp_base_concentracion.png}
    \caption{Concentracion total de celdas vivas (base \(N_0=3, P=0.5\)).}
    \label{fig:conc-base}
\end{figure}

\begin{figure}[ht]
    \centering
    \includegraphics[width=0.48\textwidth]{figuras/exp_base_sustrato.png}
    \caption{Concentracion media de sustrato (base \(N_0=3, P=0.5\)).}
    \label{fig:sustrato}
\end{figure}

\subsection{Inhibicion espacial}
\begin{figure}[ht]
    \centering
    \includegraphics[width=0.48\textwidth]{figuras/exp_n0_curvas.png}
    \caption{Curvas de concentracion para \(N_0=\{2,3,4\}\) con \(P=0.5\).}
    \label{fig:n0}
\end{figure}

\subsection{Probabilidad de division}
\begin{figure}[ht]
    \centering
    \includegraphics[width=0.48\textwidth]{figuras/exp_prob_curvas.png}
    \caption{Curvas de concentracion para \(P=\{0.5,0.25,0.125\}\) con \(N_0=3\).}
    \label{fig:prob}
\end{figure}

\subsection{Efecto del sustrato inicial}
\begin{figure}[ht]
    \centering
    \includegraphics[width=0.48\textwidth]{figuras/exp_sustrato_curvas.png}
    \caption{Curvas de concentracion para \(s_0=\{60,70\}\,\si{g/L}\) con \(N_0=3, P=0.5\).}
    \label{fig:s0}
\end{figure}

\subsection{Comparacion con modelo cinetico}
\begin{figure}[ht]
    \centering
    \includegraphics[width=0.48\textwidth]{figuras/exp_cinetico.png}
    \caption{Curva CA vs. modelo cinetico logistico simple.}
    \label{fig:cinetico}
\end{figure}

\section{Discusion}
Las simulaciones reproducen los hallazgos cualitativos del articulo: menor \(N_0\) enlentece la expansion, probabilidades menores retrasan el frente, y mayor \(s_0\) acelera la densificacion. La curva cinetica simple se alinea con la CA en tendencia general. Las animaciones permiten explorar las variantes en vivo sin relanzar el script.

\section{Conclusiones}
El CA implementado capta la inhibicion espacial y la limitacion de sustrato. El pipeline genera automaticamente las figuras del articulo (\texttt{python main.py}) y ofrece un visor animado para demostraciones (\texttt{python animacion.py}).

\end{document}
