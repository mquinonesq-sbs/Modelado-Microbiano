\documentclass[10pt,conference]{IEEEtran}
\usepackage[utf8]{inputenc}
\usepackage{graphicx}
\usepackage{amsmath}
\usepackage{amsfonts}
\usepackage{siunitx}
\usepackage{hyperref}
\usepackage{caption}
\usepackage{subcaption}
\usepackage{cite}

\title{Modelado y simulación del crecimiento microbiano mediante autómatas celulares bidimensionales}
\author{Codex GPT-5\\Trabajo-Maye}

\begin{document}
\maketitle

\begin{abstract}
Se implementó un autómata celular (CA) bidimensional para reproducir y extender el modelo de crecimiento microbiano descrito en el artículo de referencia. El modelo emplea una malla de \num{200x200} celdas, vecindario de Moore, umbral de inhibición espacial \(N_0\), probabilidades de división \(P_1=0.5\), \(P_2=0.25\), \(P_3=0.125\), y una dinámica acoplada de sustrato. Se evaluó el efecto de la limitación de sustrato y la inhibición espacial sobre la propagación celular. Las simulaciones (300 iteraciones) producen distribuciones espaciales y curvas de concentración comparables con las Fig. 3--5 del artículo.\end{abstract}

\section{Introducción}
Los autómatas celulares (CA) permiten estudiar la evolución microbiana mediante reglas locales simples que capturan fenómenos de crecimiento, división y muerte. El artículo de referencia plantea un CA 2D con estados vacíos, división y crecimiento, vecindario de Moore de 8 vecinos, y un umbral de inhibición espacial \(N_0\) que controla la transición entre división y crecimiento. Además, la disponibilidad de sustrato limita la expansión. Aquí implementamos el modelo y generamos todas las figuras solicitadas.

\section{Modelo de autómata celular}
\subsection{Espacio y estados}
El espacio es una cuadrícula \(L_i\times L_j = 200\times200\). Cada celda \(\text{Cell}(i,j)\) tiene estado \(S_{ij}(t) \in \{0,1,2\}\): 0 = vacío/muerta, 1 = división, 2 = crecimiento. El vecindario es de Moore (8 celdas adyacentes).

\subsection{Efecto de inhibición espacial}
Sea \(N_{1}(i,j,t)\) el número de vecinos en estado 1 y \(N_{2}(i,j,t)\) en estado 2. El umbral \(N_0\) refleja el efecto inhibidor del espacio.
\begin{equation}
\text{Si } S_{ij}(t)=1 \text{ y } N_{1}+N_{2}>N_0 \Rightarrow S_{ij}(t+1)=2
\end{equation}
Cuando \(N_{1}+N_{2}\leq N_0\), la célula en división puede dividirse con probabilidad \(P_j\), creando una célula hija en un vecino vacío y pasando la madre a crecimiento.

\subsection{Efecto de limitación de sustrato}
El sustrato inicial es uniforme \(s(0)=60\,\si{g/L}\). Cada paso incluye difusión (promedio de vecinos con coeficiente \(D=0.1\)) y consumo: \(c_1\) por celda en división y \(c_2\) por celda en crecimiento. Si el sustrato local cae por debajo de un umbral, se inhibe la división y puede ocurrir muerte estocástica por inanición.

\subsection{Probabilidades y parámetros}
Se evaluaron \(N_0\in\{2,3,4\}\) y \(P\in\{0.5,0.25,0.125\}\). Se simularon 300 iteraciones y se tomaron capturas en \(t=1,5,10,30\) h.

\section{Algoritmo}
\begin{enumerate}
  \item Inicializar \(S_{ij}\) aleatoriamente con mayoría de celdas vacías y sustrato uniforme.
  \item Para cada tiempo: difundir sustrato, consumir según estado.
  \item Computar \(N_1, N_2\) (vecindario de Moore).
  \item Regla de inhibición: división \(\to\) crecimiento si \(N_1+N_2>N_0\).
  \item División estocástica: si hay vecinos vacíos, con probabilidad \(P\) se ocupa uno con estado 1 y la madre pasa a 2.
  \item Crecimiento \(\to\) división con probabilidad \(P\) si \(N_1+N_2\le N_0\) y hay sustrato.
  \item Muerte estocástica por inanición cuando el sustrato es muy bajo.
\end{enumerate}

\section{Resultados}
\subsection{Distribución espacial}
La Fig.~\ref{fig:espacial} muestra la expansión del cultivo para \(N_0=3\) y \(P=0.5\) en distintos tiempos.
\begin{figure}[ht]
    \centering
    \includegraphics[width=0.48\textwidth]{figuras/espacial_t1_5_10_30.png}
    \caption{Distribución espacial a \(t=1,5,10,30\) h. Azul: división, rojo: crecimiento.}
    \label{fig:espacial}
\end{figure}

\subsection{Curvas de concentración para $N_0$}
\begin{figure}[ht]
    \centering
    \includegraphics[width=0.48\textwidth]{figuras/curva_concentracion_n0.png}
    \caption{Concentración microbiana total vs. tiempo para \(N_0=2,3,4\).}
    \label{fig:n0}
\end{figure}

\subsection{Curvas de probabilidades}
\begin{figure}[ht]
    \centering
    \includegraphics[width=0.48\textwidth]{figuras/curva_probabilidades.png}
    \caption{Concentración microbiana para distintas probabilidades \(P\) con \(N_0=3\).}
    \label{fig:prob}
\end{figure}

\subsection{Evolución del sustrato}
\begin{figure}[ht]
    \centering
    \includegraphics[width=0.48\textwidth]{figuras/sustrato_base.png}
    \caption{Concentración media de sustrato (\(N_0=3\), \(P=0.5\)).}
    \label{fig:sustrato}
\end{figure}

\section{Discusión}
Se reproduce el comportamiento cualitativo reportado en el artículo: a menor \(N_0\), la inhibición espacial reduce la expansión; probabilidades menores retrasan el frente de crecimiento. La inclusión de sustrato amortigua el crecimiento inicial. El modelo es extensible a condiciones heterogéneas o parámetros distintos.

\section{Conclusiones}
El CA implementado capta los efectos de inhibición espacial y limitación de sustrato en el crecimiento microbiano. Las figuras generadas son análogas a las Fig. 3--5 del artículo y permiten explorar variaciones de \(N_0\) y \(P\).

\section*{Bibliografía}
\begin{thebibliography}{99}
\bibitem{men2013} H. Men y X. Zhao, ``Modelado y simulación del crecimiento microbiano basados en autómatas celulares,'' Revista de Investigación en Ciencias Aplicadas, Ingeniería y Tecnología, vol. 6, no. 11, pp. 2061--2066, 2013.
\bibitem{lacasta1999} A. Lacasta et al., ``Modelado de patrones espaciotemporales en colonias bacterianas,'' Phys. Rev. E, 59:7036--7041, 1999.
\end{thebibliography}

\end{document}
