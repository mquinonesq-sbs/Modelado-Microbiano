\documentclass[10pt]{article}
\usepackage[utf8]{inputenc}
\usepackage{geometry}
\geometry{margin=1in}
\title{Transcripcion del PDF ``Microbial Growth Modeling and Simulation Based on Cellular Automata''}
\author{Extracto en bruto}
\date{}
\begin{document}
\maketitle

\section*{Texto}
\begin{verbatim}
Research Journal of Applied Sciences, Engineering and Technology 6(11): 2061-2066 , 2013
ISSN: 2040-7459; e-ISSN: 2040-7467
© Maxwell Scientific Organization, 2013
Submitted: November 29, 2012 Accepted: January 17, 2013 Published: July 25, 2013

Corresponding Author: Men Hong, School of Automation Engineering, Northeast Dianli University, Jilin 132012 , China, Tel.:
+86-432-64807283
2061
Microbial Growth Modeling and Simulation Based on Cellular Automata

Hong Men and Xiaojuan Zhao
School of Automation Engineering, Northeast Dianli University, Jilin 132012, China

Abstract: In order to simulate the micro- evolutionary process of the microbial growth, [Methods] in this study, we
adopt two-dimensional cellular automata as its growth space. Based on evolutionary mechanism of microbial and
cell-cell interactions, we adopt Moore neighborhood and make the transition rules. Finally, we construct the
microbial growth model. [Results] It can describe the relationships among the cell growth, division and death. And
also can effectively reflect spatial inhibition effect and substrate limitation effect. [Conclusions] The simulation
results show that CA model is not only consistent with the classic microbial kinetic model, but also be able to
simulate the microbial growth and evolution.
Keywords: Cellular automata model, substrate limitation effect, space inhibition effect, transition rules

INTRODUCTION

The research of the evolution and the growth
process of microbial individual cells have a certain
effect on improving the microbial production and
optimize microbial culture conditions. The mathematical model is a powerful tool for microbial
growth simulation.
Classic microbial kinetics model (Chunrong, 2004)
utilizes continuous differential equations, microbial
growth utilizes Monod equation. Assumed microbial
growth rate is proportional to the substrate consumption
rate which can describe microbial growth and substrate
removal kinetics from a macro point of view. Once the
growth of microorganisms involves micro issues, such
as mechanisms of microbial evolution, microbial
characteristics (diversity, randomness, sensitivity) in
complex systems, we must consider it from microscopic
aspect. But currently, the existed computer models
focus on the analysis from microscopic perspective, such as discrete models (Saadia and Marie, 2002),
communications walking model (Fogedby, 1991) which
based on the nutrient diffusion-controlled growth. They
consider less factors and the research scope is limited.
CA model (Heiko et al., 1998) is based on a very
limited set of rules but it shows the diversity of
microbial growth behavior (microbial cell growth, decline and fall off, etc.). It bases on qualitative
conclusions of the biology to design simple local evolution rules to examine the characteristics of microorganisms, which offers a theoretical CA model for the
modeling and simulation of the complex system. At the same time, it has special meaning for the research of the
micro evolution process of microorganisms in the
complex system. This study establishes the CA model of microbial
growth and simulates the process of microbial growth
evolution. It also makes a research on the influence of the substrate limit effect and space inhibition effect to
the evolution process.

... (continua la transcripción en el PDF original)
\end{verbatim}

\end{document}
